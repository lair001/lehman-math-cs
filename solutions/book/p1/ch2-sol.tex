\documentclass{article}
\usepackage[utf8]{inputenc}
\usepackage{indentfirst, hyperref, gensymb, amsmath, amsthm, wasysym, amsfonts, mathtools, braket, amssymb}
\hypersetup{colorlinks,allcolors=blue,linktoc=all}

\title{Chapter 2 Problem Solutions}
\author{Samuel Lair}
\date{August 2022}

\begin{document}

\maketitle
\tableofcontents

\pagebreak

\section{Problem 1}
This problem is awkwardly formulated as filling in the blanks of a proof. I'm going to simply reproduce the entire proof with the blanks filled in.

Prove that every amount of postage that can assembled using only 10 cent and 15 stamps is divisible by 5.
\begin{proof}
	Let the notation $``j | k"$ indicate that integer $j$ is a divisor of integer $k$, and let $S(n)$ mean that exactly $n$ cents postage can be assembled using only 10 and 15 cent stamps. Then we need to prove:
	\begin{equation}\label{eq:one}
		S(n) \text{ IMPLIES } 5 \mid n \text{, for all nonnegative integers }n
	\end{equation}
	Let $C$ be the set of \textit{counterexamples} to \eqref{eq:one}, namely
	\[
		C \Coloneqq \Set{n | S(n) \text{ and NOT}(5 \mid n)}
	\]
	Assume for the purpose of obtaining a contradiction that $C$ is nonempty. Then by the WOP, there is a smallest $m \in C$. This m must be positive because \eqref{eq:one} holds for $n = 0$.

	But if $S(m)$ holds and $m$ is positive, then $S(m - 10)$ or $S(m-15)$ must hold because there exist nonnegative integers $i$ and $j$ such that $m = 10i + 15j$ and at least one of $i$, $j$ is greater than zero. If $i > 0$, then $S(m - 10)$ must hold. If $j > 0$, then $S(m - 15)$ must hold.

	So suppose $S(m - 10)$ holds. Then $5 \mid (m - 10)$ because $m - 10 = 10i + 15j$ for nonnegative integers $i$ and $j$ and both $5 \mid 10$ and $5 \mid 15$ are true.

	But if $5 \mid (m - 10)$, then $5 \mid m$ because $m = 10(i+1) + 15j$, contradicting the fact that $m$ is a counterexample.

	Next, if $S(m - 15)$ holds, we arrive at a contradiction in the same way.

	Since we get a contradiction in both cases, we conclude that $C$ is empty, which proves that \eqref{eq:one} holds.
\end{proof}

\pagebreak

\section{Problem 2} At the very least, the conclusion in line 12 is wrong. Besides $F(0) = 0$, there is a second Fibonacci number whose value does not depend on two prior Fibonacci numbers: $F(1) = 1$. The proof should have considered $F(1) = 1$ after step 6, determined that $m = 1$ and concluded that $EF(n)$ does not hold for all $n \in \mathbb{N}$.

\pagebreak

\section{Problem 3}
The WOP does not apply to the positive rational numbers. To see this, simply construct an infinite subset of the positive rational numbers that asymptotically converges to zero. E.g.
\[
	S \Coloneqq \Set{\frac{1}{10}, \frac{1}{100}, \frac{1}{1000}, ... }
\]
Elements of $S$ can be generated indefinitely by multiplying the previous element by $\frac{1}{10}$ and each new element is smaller than the element preceding it. Therefore, $S$ is a subset of the positive rational numbers that does not have a minimum element.

With this in mind, the proof in question is incorrect when it makes the claim: ``So there must be a a smallest rational $q_0 \in C$." Everything past this false claim is ``bogus".
\pagebreak

\section{Problem 4}
Prove that
\begin{equation}\label{eq:two}
	\sum_{k=0}^n k^2 = \frac{n(n+1)(2n+1)}{6}
\end{equation}
for all nonnegative integers n.
\begin{proof}
	We use proof by contradiction. Suppose that the claim is false. Then, some nonnegative integers serve as \textit{counterexamples} to it. Let's collect them in a set:
	\[
		C \Coloneqq \Set{n \in \mathbb{N} | \sum_{k=0}^n k^2 \neq \frac{n(n+1)(2n+1)}{6}}
	\]
	By the WOP, $C$ has a minimum element $c$. I.e. \eqref{eq:two} is false for $n = c$ but true for all nonnegative integers $n < c$. Since \eqref{eq:two} is true for $n = 0$, $c > 0$.  Therefore, $c - 1$ is a nonnegative integer for which \eqref{eq:two} holds:
	\[
		\sum_{k=0}^{c-1} k^2 = \frac{(c-1)(c-1+1)(2(c-1)+1)}{6} = \frac{c(c-1)(2c-1)}{6}
	\]
	But if we add $c^2$ to both sides of this equation, we find:
	\begin{align*}
		\sum_{k=0}^c k^2 = & \frac{c(c-1)(2c-1)}{6} + c^2 = \frac{(c^2-c)(2c-1)+6c^2}{6} \\ =& \frac{2c^3 - c^2 - 2c^2 + c + 6c^2}{6} = \frac{c(2c^2+3c+1)}{6} = \frac{c(c+1)(2c+1)}{6}
	\end{align*}
	Therefore, we've reached a contradiction where \eqref{eq:two} holds for $c$. Hence, $C$ is empty and the claim must be true.
\end{proof}

\pagebreak

\section{Problem 5}
Prove that
\begin{equation}\label{p5claim}
	\nexists \; a, b, c \in \mathbb{Z^+} \text{ such that } 4a^3 + 2b^3 = c^3
\end{equation}
$\mathbb{Z^+}$ denotes the positive integers, which are a subset of $\mathbb{N}$.

\begin{proof}
	The proof is by the WOP. Let $C$ be the set of counterexamples to \eqref{p5claim}. We will prove by contradiction that $C$ is empty. Suppose that $C$ is not empty. Then by the WOP, there is a tuple $(a_0, b_0, c_0)$ among the elements of $C$ such that $a_0$ is the least value of $a$ of all such tuples in $C$. Since $4a_0^3 + 2b_0^3 = c_0^3$, $c_0^3$ is a multiple of $3$. By the $\textit{Fundamental Theorem of Arithmetic}$, $c_0$ is also a multiple of $2$. I.e., $\exists \; k \in \mathbb{Z^+}$ such that $c_0 = 2k$. Then:
	\[
		2b_0^3 = 8k^3 - 4a_0^3 \implies b_0^3 = 4k^3 - 2a_0^3
	\]
	Therefore, $b_0^3$ is a multiple of 2. By the $\textit{Fundamental Theorem of Arithmetic}$, $b_0$ is also a multiple of $2$. I.e., $\exists \; j \in \mathbb{Z^+}$ such that $b_0 = 2j$. Continuing one more time:
	\[
		4a_0^3 = 8k^3 - 16j^3 \implies a_0^3 = 2k^3 - 4j^3
	\]
	Therefore, $a_0^3$ is a multiple of 2. By the $\textit{Fundamental Theorem of Arithmetic}$, $a_0$ is also a multiple of $2$. I.e., $\exists \; i \in \mathbb{Z^+}$ such that $a_0 = 2i$. But then:
	\[
		32i^3 + 16j^3 = 8k^3 \implies 4i^3 + 2j^2 = k^3
	\]
	Therefore, $(i, j, k) \in C$ and we have the contradiction that $i < a_0$ since $a_0 = 2i$. Hence, $C$ must be empty, implying that \eqref{p5claim} must be true.

\end{proof}

\pagebreak

\section{Problem 6}
This problem is actually about the unsigned binary representation of unsigned integers. Namely, we asked to prove that $m + 1$ unsigned bits can represent any nonnegative integer less than or equal to $2^{m + 1} - 1$. Let's restate the claim in a more straightforward manner.

Given $S \Coloneqq \Set{0, 1}$ and $n \in \mathbb{N}$ such that $n < 2^{m + 1}$ for some $m \in \mathbb{N}$, prove that:
\begin{equation}\label{p6claim}
	\exists \; a_0, a_1, ..., a_m \in S \text{ such that } n = a_0 2^0 + a_1 2^1 + ... + a_m 2^m
\end{equation}

\begin{proof}
	The proof is by case analysis. Consider the following cases:
	\begin{enumerate}
		\item $m = 0$
		\item $m > 0$
	\end{enumerate}

	\textbf{Case 1:}
	We need to consider nonnegative integers less than $2^{0 + 1} = 2$. Fortunately, there are only two such nonnegative integers: 0 and 1. Consider that:
	\[
		0 = 0 \cdot 2^0
	\]
	\[
		1 = 1 \cdot 2^0
	\]
	\eqref{p6claim} holds for Case 1.

	\textbf{Case 2:}
	The proof for this case is by the WOP. Let $C$ be the set of counterexamples to \eqref{p6claim}. We will prove by contradiction that $C$ is empty. Suppose that $C$ is not empty. By the WOP, there is a least integer $q \in C$. Since $q$ is the least counterexample, \eqref{p6claim} must hold for $q - 1$:
	\[
		q - 1 = a_0 2^0 + a_1 2^1 + ... + a_m 2^m
	\]
	\[
		q = a_0 2^0 + a_1 2^1 + ... + a_m 2^m + 1
	\]

	Let us now consider two subcases:
	\begin{enumerate}
		\item $q - 1 < 2^{m + 1} - 1$
		\item $q - 1 = 2^{m + 1} - 1$
	\end{enumerate}

	\textbf{Case 2.1:}
	Since $q - 1 < 2^{m + 1} - 1$, at least one of the coefficients $a_0, ..., a_m$ equals 0. By the WOP, there exists a least index i such that $a_i = 0$. Then:
	\[
		q = q - 1 + 1 = 0 \cdot 2^0 + 0 \cdot 2^1 + ... + 1 \cdot 2^i + a_{i+1} \cdot 2^{i+1} + ... + a_m \cdot 2^m
	\]
	This leads us to the contradiction that \eqref{p6claim} holds for q. Therefore, $C$ is empty and \eqref{p6claim} holds in this subcase.

	\textbf{Case 2.2:}
	In this case, $q = q - 1 + 1 = 2^{m+1} - 1 + 1 = 2^{m + 1}$. Therefore, we have a contradiction where $q$ isn't a counterexample since \eqref{p6claim} only makes a claim for $n < 2^{m + 1}$. Therefore, $C$ is empty and \eqref{p6claim} holds in this subcase.

	$C$ is empty and \eqref{p6claim} holds in Subcases 2.1 and 2.2. Therefore, $C$ is empty and \eqref{p6claim} holds in Case 2.

	\eqref{p6claim} holds for Cases 1 and 2. Therefore, \eqref{p6claim} must be true.

\end{proof}

\pagebreak

\section{Problem 7}
Prove that any integer greater than or equal to 8 can be represented as the sum of nonnegative integer multiples of 3 and 5.
\begin{proof}
	Our claim can be restated as follows:
	\begin{equation}\label{p7claim}
		\forall \; n \in \mathbb{N}, \exists \; i, j \in \mathbb{N} \text{ such that } n + 8 = 3i + 5j
	\end{equation}

	The proof is by the WOP. Let $C$ be the set of counterexamples to \eqref{p7claim}. We will prove by contradiction that $C$ is empty. Suppose that $C$ is not empty. By the WOP, there is a least nonnegative integer $m \in C$. Consider the following:
	\[
		0 + 8 = 8 = 3 + 5
	\]
	\[
		1 + 8 = 9 = 3 \cdot 3
	\]
	\[
		2 + 8 = 10 = 2 \cdot 5
	\]
	Therefore, $m \geq 3$, which ensures that $(m - 3) \in \mathbb{N}$. Since m is the minimum counterexample, \eqref{p7claim} must hold for $m - 3$. I.e.
	\[
		\exists \; i, j \in \mathbb{N} \text{ such that } m - 3 + 8 = 3i + 5j
	\]
	But then $m + 8 = 3(i+1) + 5j \implies m \notin C$, a contradiction. Therefore, $C$ is empty. Hence, \eqref{p7claim}, which is equivalent to our claim, is true.
\end{proof}

\pagebreak

\section{Problem 8}
Prove that any integer greater than or equal to
50 can be represented as the sum of nonnegative integer multiples of 7, 11, and 13.

\begin{proof}
	Our claim can be restated as follows:
	\begin{equation}\label{p8claim}
		\forall \; n \in \mathbb{N}, \exists \; i, j, k \in \mathbb{N} \text{ such that } n + 50 = 7i + 11j + 13k
	\end{equation}

	The proof is by the WOP. Let $C$ be the set of counterexamples to \eqref{p8claim}. We will prove by contradiction that $C$ is empty. Suppose that $C$ is not empty. By the WOP, there is a least nonnegative integer $m \in C$. Consider the following:
	\[
		0 + 50 = 50 = 1 \cdot 11 + 3 \cdot 13
	\]
	\[
		1 + 50 = 51 = 2 \cdot 7 + 1 \cdot 11 + 2 \cdot 13
	\]
	\[
		2 + 50 = 52 = 4 \cdot 13
	\]
	\[
		3 + 50 = 53 = 2 \cdot 7 + 3 \cdot 13
	\]
	\[
		4 + 50 = 54 = 4 \cdot 7 + 2 \cdot 13
	\]
	\[
		5 + 50 = 55 = 5 \cdot 11
	\]
	\[
		6 + 50 = 56 = 8 \cdot 7
	\]
	Therefore, $m \ge 7$, which ensures that $(m - 7) \in \mathbb{N}$. Since $m$ is the minimum counterexample, \eqref{p8claim} must hold for $m - 7$. I.e.
	\[
		\exists \; i, j, k \in \mathbb{N} \text{ such that } m - 7 + 50 = 7i + 11j + 13k
	\]
	But then $m + 50 = 7(i+1) + 11j + 13k \implies m \notin C$, a contradiction. Therefore, $C$ is empty and \eqref{p8claim}, which is equivalent to our claim, must be true.
\end{proof}

\pagebreak

\section{Problem 9}
Prove that
\begin{equation}\label{p9claim}
	\nexists \; a, b, c, d \in \mathbb{Z^+} \text{ such that } 8a^4 + 4b^4 + 2c^4 = d^4.
\end{equation}

\begin{proof}
	The proof is by the WOP. Let $C$ be the set of counterexamples to \eqref{p9claim}. We will prove by contradiction that $C$ is empty. Suppose that $C$ is not empty. Then by the WOP, there is a tuple $(a_0, b_0, c_0, d_0)$ among the elements of $C$ such that $a_0$ is the least value of $a$ of all such tuples in $C$. Since $8a_0^4 + 4a_0^4 + 2c_0^4 = d_0^4$, $d_0^4$ is a multiple of $2$. By the \textit{Fundamental Theorem of Arithmetic}, $d_0$ is also a multiple of $2$. I.e. $\exists \; k \in \mathbb{Z^+}$ such that $d_0 = 2k$. Then:
	\[
		c_0^4 = 8k^4 - 2b_0^4 - 4a_0^4
	\]
	Therefore, $c_0^4$ is a multiple of $2$. By the \textit{Fundamental Theorem of Arithmetic}, $c_0$ is also a multiple of $2$. I.e. $\exists \; j \in \mathbb{Z^+}$ such that $c_0 = 2j$. Then:
	\[
		b_0^4 = 4k^4 - 8j^4 - 2a_0^4
	\]
	Therefore, $b_0^4$ is a multiple of $2$. By the \textit{Fundamental Theorem of Arithmetic}, $b_0$ is also a multiple of $2$. I.e. $\exists \; i \in \mathbb{Z^+}$ such that $b_0 = 2i$. Then:
	\[
		a_0^4 = 2k^4 - 4j^4 - 8i^4.
	\]
	Therefore, $a_0^4$ is a multiple of $2$. By the \textit{Fundamental Theorem of Arithmetic}, $a_0$ is also a multiple of $2$. I.e. $\exists \; h \in \mathbb{Z^+}$ such that $a_0 = 2h$. But then:
	\[
		128h^4 + 64i^4 + 32j^4 = 16k^4 \implies 8h^4 + 4i^4 + 2^4 = k^4
	\]
	Therefore, $(h, i, j, k) \in C$. However, this is contradiction since $a_0 = 2h \implies h < a_0$. Hence, $C$ is empty and \eqref{p9claim} must be true.

\end{proof}

\pagebreak

\section{Problem 10}
Prove that
\begin{equation}\label{p10claim}
	n \leq 3^{\frac{n}{3}} \; \forall \; n \in \mathbb{N}
\end{equation}
\begin{proof}
	The proof is by case analysis. Consider the following cases:
	\begin{enumerate}
		\item $n = 0$
		\item $n = 1$
		\item $n = 2$
		\item $n = 3$
		\item $n = 4$
		\item $n \geq 5$
	\end{enumerate}

	\textbf{Case 1:}
	\[
		3^{\frac{0}{3}} = 3^0 = 1 > 0
	\]
	\eqref{p10claim} holds for Case 1.

	\textbf{Case 2:}
	\[
		3^{\frac{1}{3}} = 3^{\frac{1}{3}} > 1.4422 > 1
	\]
	\eqref{p10claim} holds for Case 2.

	\textbf{Case 3:}
	\[
		3^{\frac{2}{3}} = 3^{\frac{2}{3}} > 2.0800 > 2
	\]
	\eqref{p10claim} holds for Case 3.

	\textbf{Case 4:}
	\[
		3^{\frac{3}{3}} = 3
	\]
	\eqref{p10claim} holds for Case 4.

	\textbf{Case 5:}
	\[
		3^{\frac{4}{3}} > 4.3267 > 4
	\]
	\eqref{p10claim} holds for Case 5

	\textbf{Case 6:}
	The proof for this case is by the WOP. Let $C$ be the set of counterexamples to \eqref{p10claim}. We will prove by contradiction that $C$ is empty. Suppose that $C$ is not empty. By the WOP, there is a least $m \in C$. Since \eqref{p10claim} holds in Cases 1-5, we know that $m \geq 5$ and $m - 4 \in \mathbb{N}$. Since $m$ is the least element in $C$, \eqref{p10claim} must hold for $m - 1$:
	\[
		m - 1 \leq 3^{\frac{m - 1}{3}}
	\]
	\[
		(m - 1)^3 \leq 3^{m-1}
	\]
	\[
		3 \cdot (m - 1)^3 \leq 3^m
	\]
	\[
		\text{Since } m \in C, m > 3^{\frac{m}{3}} \implies 3^m < m^3
	\]
	\[
		3 \cdot (m - 1)^3 \leq 3^m < m^3
	\]
	\[
		\frac{(m - 1)^3}{m^3} < \frac{1}{3}
	\]
	\[
		\frac{m - 1}{m} < \frac{1}{3^{\frac{1}{3}}}
	\]
	\[
		1 - \frac{1}{m} < \left(\frac{1}{3}\right)^{\frac{1}{3}}
	\]
	\[
		\frac{1}{m} > 1 - \left(\frac{1}{3}\right)^{\frac{1}{3}}
	\]
	\[
		m < \frac{1}{1 - \left(\frac{1}{3}\right)^{\frac{1}{3}}} < 3.2612
	\]
	This is a contradiction since we established by Cases 1-5 that $m \ge 5$. Therefore, $C$ is empty and \eqref{p10claim} holds for Case 6.

	\eqref{p10claim} holds for Cases 1-6. Therefore, \eqref{p10claim} must be true.
\end{proof}

\pagebreak

\section{Problem 11}
We will denote \textit{winning configurations} as lists of the form $[a, b, c, ...]$ where $a$ represents a $2 \times 2$ piece, $b$ represents a $1 \times 2$ piece, and $c$ represents a $2 \times 1$ piece. The ordering of the list is determined by scanning each row of the board from left to right.

\subsection{(a)}
For $n = 1$, the only \textit{winning configuration} is $[c]$ since the $2 \times 1$ piece is the only piece that will fit on a $2 \times 1$ board and it completely covers the board. Therefore, $T_1 = 1$.

For $n = 2$, the \textit{winning configurations} are $[a]$, $[b, b]$, and $[c, c]$. Therefore, $T_2 = 3$.

For $n = 3$, the \textit{winning configurations} are $[a, c]$, $[c, a]$, $[c, c, c]$, $[b, b, c]$, and $[c, b, b]$. Therefore, $T_3 = 5$.

\subsection{(b)}
From the very limited data we have in part (a), it appears that:
\begin{equation}\label{p11recurrence}
	T_n = T_{n - 1} + 2 \cdot T_{n - 2}
\end{equation}

\subsection{(c)}
Prove that the number of \textit{winning configurations} $T_n$ on a $2 \times n$ Mini-Tetris board is:
\begin{equation}\label{p11claim}
	T_n = \frac{2^{n + 1} + (-1)^n}{3} \text{ where } n \in \mathbb{Z^+}
\end{equation}

\begin{proof}
	The proof is by case analysis. Consider the following cases:
	\begin{enumerate}
		\item $n = 1$
		\item $n = 2$
		\item $n = 3$
		\item $n \ge 4$
	\end{enumerate}

	\textbf{Case 1:}
	\[
		\frac{2^{1 + 1} + (-1)^1}{3} = \frac{2^2 - 1}{3} = \frac{3}{3} = 1
	\]
	\eqref{p11claim} holds for Case 1.

	\textbf{Case 2:}
	\[
		\frac{2^{2 + 1} + (-1)^2}{3} = \frac{2^3 + 1}{3} = \frac{9}{3} = 3
	\]
	\eqref{p11claim} holds for Case 2.

	\textbf{Case 3:}
	\[
		\frac{2^{3 + 1} + (-1)^3}{3} = \frac{2^4 - 1}{3} = \frac{15}{3} = 5
	\]
	\eqref{p11claim} holds for Case 3.

	\textbf{Case 4:}
	The proof for this case is by the WOP. Let $C$ be the set of counterexamples to \eqref{p11claim}. We will prove by contradiction that $C$ is empty. Suppose that $C$ is not empty. By the WOP, there is a least $m \in C$. From Cases 1-3, we know that $m \geq 4$ and $(m - 3) \in \mathbb{Z^+}$. Since $m$ is the least element of $C$, \eqref{p11claim} must hold for $m - 1$ and $m - 2$. From \eqref{p11recurrence}, we find that:
	\[
		T_m = \frac{2^m + (-1)^{m - 1}}{3} + 2 \cdot \frac{2^{m-1} + (-1)^{m-2}}{3}
	\]
	\[
		T_m = \frac{2^m + 2^m + (-1)^{m - 1} + 2 \cdot (-1)^{m-2}}{3}
	\]
	\[
		T_m = \frac{2^{m + 1} + (-1)^{m - 2}}{3}
	\]
	\[
		T_m = \frac{2^{m + 1} + (-1)^m}{3}
	\]
	Therefore, we have the contradiction that \eqref{p11claim} holds for m after all! $C$ must be empty and \eqref{p11claim} holds for Case 4.

	\eqref{p11claim} holds for Cases 1-4. Hence, \eqref{p11claim} must be true.

\end{proof}

\pagebreak

\section{Problem 15}
\subsection{(a)}
The set of counterexamples to (2.8) are a subset of $\mathbb{N}$. Therefore, by the WOP, there must a least counterexample $m$.

\subsection{(b)}
Consider the following:
\[
	\sum_{i=0}^{1-1} (2i + 1) = 2 \cdot 0 + 1 = 1 = 1^2
\]
Therefore, (2.8) holds for $n = 1$. Since (2.8) is limited to $n \in \mathbb{Z^+}$, it follows that $m \ge 2$.

\subsection{(c)}
Since $m \ge 2$, $(m - 1) \in \mathbb{Z^+}$. Furthermore, since $m$ is the least counterexample, (2.8) holds for $m - 1$. Also, consider that:
\[
	\sum_{i=0}^{n-1} (2i + 1) = \sum_{i=1}^{n} (2(i-1)+1)
\]
Hence, (2.9) must hold.

\subsection{(d)}
We should add $2(m - 1) + 1$

\subsection{(e)}
Adding $2(m - 1) + 1$ to both sides of (2.9), we find:
\[
	\sum_{i=1}^m (2(i-1)+1) = (m-1)^2 + 2(m-1) + 1 = m^2 - 2m + 1 + 2m - 2 + 1 = m^2
\]
Therefore, we've reached a contradiction where (2.8) holds for $m$. The set of counterexamples must be empty and (2.8) must hold for all $n \in \mathbb{Z^+}$.

\pagebreak

\section{Problem 16}
Prove that
\begin{equation}\label{p16claim}
	\sum_{i=1}^n 2i = n(n+1) \; \forall \; n \in \mathbb{Z^+}
\end{equation}
\begin{proof}
	The proof is by the WOP. Let $C$ be the set of counterexamples to \eqref{p16claim}. We will prove by contradiction that $C$ is empty. Suppose $C$ is not empty. By the WOP, there is a least counterexample $m \in C$. Observe that \eqref{p16claim} holds for n = 1:
	\[
		\sum_{i=1}^1 2i = 2 = 1(1+1)
	\]
	Therefore, $m \ge 2$ and $(m - 1) \in \mathbb{Z^+}$. Since $m$ is the minimum counterexample, \eqref{p16claim} must hold for $m - 1$:
	\begin{equation}\label{p16mm1}
		\sum_{i=1}^{m-1} 2i = (m-1)(m-1+1) = m(m-1)
	\end{equation}
	Adding 2m to both sides of \eqref{p16mm1}, we find:
	\[
		\sum_{i=1}^m 2i = m(m-1) + 2m = m^2 -m + 2m = m^2 + m = m(m+1).
	\]
	We've reached a contradiction where \eqref{p16claim} holds for $m$. Hence, $C$ is empty and \eqref{p16claim} must be true.
\end{proof}

\pagebreak

\section{Problem 17}
Prove that
\begin{equation}\label{p17claim}
	\sum_{i=0}^n i^3 = \left( \frac{n(n+1)}{2} \right)^2 \forall n \in \mathbb{N}
\end{equation}
\begin{proof}
	The proof is by the WOP. Let $C$ be the set of counterexamples to \eqref{p17claim}. We will prove by contradiction that $C$ is empty. Suppose $C$ is not empty. By the WOP, there is a least counterexample $m \in C$. Observe that \eqref{p17claim} holds for $n \in \{0, 1, 2\}$:
	\begin{align*}
		 & \left( \frac{0(0+1)}{2} \right)^2 = 0 = 0^3 = \sum_{i=0}^0 i^3                   \\
		 & \left( \frac{1(1+1)}{2} \right)^2 = 1^2 = 1 = 0^3 + 1^3 = \sum_{i=0}^1 i^3       \\
		 & \left( \frac{2(2+1)}{2} \right)^2 = 3^2 = 9 = 0^3 + 1^3 + 2^3 = \sum_{i=0}^2 i^3
	\end{align*}
	Therefore, $m \ge 2$ and $(m - 1) \in \mathbb{N}$. Since $m$ is the least counterexample, \eqref{p17claim} must hold for $m - 1$:
	\begin{equation}\label{p17mm1}
		\sum_{i=0}^{m-1} i^3 = \left( \frac{(m-1)(m-1+1)}{2} \right)^2 = \left( \frac{m(m-1)}{2} \right)^2
	\end{equation}
	Adding $m^3$ to both sides of \eqref{p17mm1}, we find:
	\begin{align*}
		\sum_{i=0}^m i^3 & = \left( \frac{m(m-1)}{2} \right)^2 + m^3 = \frac{m^4 - 2m^3 + m^2 + 4m^3}{4} = \frac{m^4 + 2m^3 + m^2}{4} \\ &= \left( \frac{m(m+1)}{2} \right)^2
	\end{align*}
	We've reached a contradiction where \eqref{p17claim} holds for $m$. Hence, $C$ is empty and \eqref{p17claim} must be true.
\end{proof}

\pagebreak

\section{Problem 21}

\subsection{(a)}
Observe that:
\[
	S \Coloneqq \Set{n \in \mathbb{Z} | n \geq -\sqrt{2}} = \Set{n \in \mathbb{Z} | n \geq -1}
\]
It follows from Theorem 2.4.1 on page 33 of the book that $S$ is well ordered.

\subsection{(b)}
Let
\[
	S \Coloneqq \Set{r \in mathbb{R} | r \geq \sqrt{2}}
\]
\[
	T \Coloneqq \Set{r \in \mathbb{R} | r = \sqrt{2} + \frac{1}{n} \; \forall \; n \in \mathbb{Z^+}}
\]
S is not well ordered. T is a subset of S that has no minimum element.

\subsection{(c)}
Let
\[
	S \Coloneqq \Set{q \in \mathbb{Q} | q = \frac{1}{n} \; \forall \; n \in \mathbb{Z^+}}
\]
$S$ in not well ordered since it does not have a minimum element.

\subsection{(d)}
$G$ is well ordered since there is an upper bound of $10^{100}$ on the denominator $n$. I.e. every subset $S$ of $G$ has a one-to-one mapping to a subset of $\mathbb{N}$ obtained by multiplying each element of $S$ by $10^{100}!$.

\subsection{(e)}
$\mathbb{F}$ is discussed in section 2.4.1 of the book where it is established to be well ordered. The least element of a subset of $\mathbb{F}$ is the element with the smallest numerator.  Such a smallest numerator is guaranteed to exist by the WOP since $n \in \mathbb{N}$.

\subsection{(f)}
Let
\begin{align*}
	S \Coloneqq & 1, \frac{3}{4}, \frac{2}{3}, \frac{1}{2}, \frac{0}{1} \\
	T \Coloneqq & 1, \frac{48}{49}, ..., \frac{1}{2}, \frac{0}{1}       \\
	U \Coloneqq & 1, \frac{498}{499}, ..., \frac{1}{2}, \frac{0}{1}
\end{align*}
$S$ is a length 5 decreasing sequence of elements of $W$ starting with 1. $T$ is a length 50 decreasing sequence of elements of $W$ starting with 1. $U$ is a length 500 decreasing sequence of elements of $W$ starting with 1.

\pagebreak

\section{Problem 22}
Prove that every finite, a nonempty set of real numbers has a minimum element.
\begin{proof}
	The proof is by the WOP. Let $C$ be the set of sizes of sets of real numbers that are counterexamples to our claim. I.e. $n \in C \iff \exists T \subset \mathbb{R}$ such that $T$ is a counterexample to our claim and $T$ has exactly $n$ elements.  We will prove by contradiction that $C$ is empty.

	Suppose $C$ is not empty. By the WOP, $C$ has a least element $m$. Our claim obviously holds for $n = 1$ since the minimum element of a size 1 set of real numbers is its sole element. Therefore, $m > 1$ and $(m - 1) \in \mathbb{Z^+}$.

	Since $m$ is the least element of $C$, our claim must hold for $m - 1$.  However, any set $S$ of size $m$ contains all of the elements of a set $U$ of size $m - 1$ plus an additional element $r_m$. Let $r_0$ be the minimum element of the $U$.  Then the minimum element of $S$ must be the minimum of $r_m$ and $r_0$. This is a contradiction. Hence, $C$ is empty and our claim must be true.
\end{proof}

\pagebreak

\section{Problem 23}
Prove that a set $R$ of real numbers is well ordered iff there is no finite decreasing sequence of numbers in $R$. In other words, there is no set $T$ of numbers $r_i \in R$ such that
\begin{equation}\label{p23decreasing}
	r_0 > r_1 > r_2 > ... > r_n > r_{n+1} > ...
\end{equation}

\begin{proof}
	Since we have an iff claim, there are two parts of the claim we must prove:
	\begin{enumerate}
		\item If a set $R$ of real numbers is well ordered then there is no infinite decreasing sequence $S$ of numbers in $R$.
		\item If there is no infinite decreasing sequence $S$ of numbers in a set $R$ of real numbers, then $R$ is well ordered.
	\end{enumerate}

	\textbf{Part 1:}
	The proof is by contradiction. Suppose $R$ is well ordered but $S$ does exist. Then the corresponding set $T$ would be a subset of $R$. However, $T$ does not have a minimum element since $r_{n+1} < r_n \; \forall \; r_n \in T$.  Therefore, $R$ is not well ordered, a contradiction. Hence, Part 1 of the claim must be true.

	\textbf{Part 2:}
	The proof is by contradiction. Suppose $S$ does not exist but $R$ is not well ordered. Then $R$ has a subset $T$ that does not have a minimum a minimum element. But then $S$ exists after all since we can construct it from the elements of $T$. This is a contradiction. Hence, Part 2 of the claim must be true.

	Parts 1 and 2 of the claim are true. Hence, the claim must be true.
\end{proof}

\end{document}
