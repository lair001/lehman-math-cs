\documentclass{article}
\usepackage[utf8]{inputenc}
\usepackage[T1]{fontenc}
\usepackage{indentfirst, hyperref, gensymb, amsmath, amsthm, wasysym, amsfonts, mathtools, braket, amssymb}
\hypersetup{colorlinks,allcolors=blue,linktoc=all}

\title{In-Class Problem Solutions for Session 6}
\author{Samuel Lair}
\date{September 2022}

\begin{document}

\maketitle
\tableofcontents

\pagebreak

\section{Problem 1}
\subsection{(a)}
Prove that
\begin{equation}\label{p3aclaim}
	(P \land \neg Q) \lor (P \land Q) \equiv P
\end{equation}
\begin{proof}
	The proof is by truth table.

	\[
		\begin{array}{|c|c|c c c|}
			P          & Q & (P \land \neg Q) & \lor       & (P \land Q) \\
			\hline
			\mathbf{T} & T & F                & \mathbf{T} & T           \\
			\mathbf{T} & F & T                & \mathbf{T} & F           \\
			\mathbf{F} & T & F                & \mathbf{F} & F           \\
			\mathbf{F} & F & F                & \mathbf{F} & F
		\end{array}
	\]

	The truth table for $(P \land \neg Q) \lor (P \land Q)$ is identical to the truth table for $P$. Hence, \eqref{p3aclaim} must be true.
\end{proof}

\subsection{(b)}
Prove that
\begin{equation}\label{p3bclaim}
	A = (A - B) \cup (A \cap B)
\end{equation}

\begin{proof}
	\begin{align*}
		 & x \in (A - B) \cup (A \cap B)                           & \iff                                                                                         \\
		 & x \in (A - B) \lor x \in (A \cap B)                     & \iff                                                                                         \\
		 & (x \in A \land x \notin B) \lor (x \in A \land x \in B) & \iff                                                                                         \\
		 & x \in A                                                 & \text{ (apply } \eqref{p3aclaim} \text{ taking } P \Coloneqq & x \in A, Q \Coloneqq x \in B) \\
	\end{align*}

	Hence, \eqref{p3bclaim} must be true.
\end{proof}

\pagebreak

\section{Problem 2}

\subsection{(a)}
\[
	(x = \emptyset) \Coloneqq \forall z.(z \notin x)
\]

\subsection{(b)}
\[
	(x = \{y, z\}) \Coloneqq \forall a.(a \in x \iff a \in \{y, z\})
\]

\subsection{(c)}
\[
	(x \subseteq y) \Coloneqq \forall z.(z \in x \implies z \in y)
\]


\subsection{(d)}
\[
	(x = y \cup z) \Coloneqq \forall a.(a \in x \iff a \in y \lor a \in z)
\]

\subsection{(e)}
\[
	(x = y - z) \Coloneqq \forall a.(a \in x \iff a \in y \land a \notin z)
\]

\subsection{(f)}
\[
	(x = \text{pow}(y)) \Coloneqq \forall z.(z \in x \iff z \subseteq y)
\]

\subsection{(g)}
\[
	(x = \bigcup_{z \in y} z) \Coloneqq \forall a.(a \in x \iff \exists z \in y.(a \in z))
\]

\pagebreak

\section{Problem 3}

\subsection{(a)}
Representing $(a, b)$ by $\{a,b\}$ doesn't work because distinct pairs would be represented by the same set. E.g. $(1,2)$ and $(2,1)$ would both be represented by $\{1,2\}$.

\subsection{(b)}
Representing $(a, b)$ by $\{a,\{b\}\}$ also doesn't work because distinct pairs would be represented by the same set. E.g. $(\{1\},2)$ and $(\{2\},1)$ would both be represented by $\{\{1\},\{2\}\}$.

\subsection{(c)}
$pair(a,b)$ uniquely determines $(a,b)$ because this particular combination of a scalar and a set element removes any ambiguity regarding which item comes first. $a$, the first item, appears in both the scalar and set elements. $b$, the second item, only appears in the set element.

\pagebreak

\section{Problem 4}
The second player still has a winning strategy when $A$ has four elements since the second player can always choose the complement of whatever subset the first player chooses on his/her first turn. Consider the following cases:
\begin{enumerate}
	\item Player 1 chooses a subset containing 1 element of $A$ on his/her first turn.
	\item Player 1 chooses a subset containing 2 elements of $A$ on his/her first turn.
	\item Player 1 chooses a subset containing 3 elements of $A$ on his/her first turn.
\end{enumerate}

\textbf{Case 1:}
Player 2 chooses the subset containing the remaining 3 elements of $A$ on his/her first turn. Player 1 has no legal moves on his/her second turn so Player 2 wins.

\textbf{Case 2:}
Player 2 chooses the subset containing the remaining 2 elements of $A$ on his/her first turn. Player 1 has no legal moves on his/her second turn so Player 2 wins.

\textbf{Case 3:}
Player 2 chooses the subset containing the remaining element of $A$ on his/her first turn. Player 1 has no legal moves on his/her second turn so Player 2 wins.

\pagebreak

\end{document}