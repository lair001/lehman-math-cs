\documentclass{article}
\usepackage[utf8]{inputenc}
\usepackage[T1]{fontenc}
\usepackage{indentfirst, hyperref, gensymb, amsmath, amsthm, wasysym, amsfonts, mathtools, braket, amssymb}
\hypersetup{colorlinks,allcolors=blue,linktoc=all}

\newcommand{\surj}{\text{ surj }}
\newcommand{\inj}{\text{ inj }}
\newcommand{\bij}{\text{ bij }}

\newenvironment{subproof}[1][\proofname]{%
	\renewcommand{\qedsymbol}{$\blacksquare$}%
	\begin{proof}[#1]%
	}{%
	\end{proof}%
}

\title{In-Class Problem Solutions for Session 7}
\author{Samuel Lair}
\date{September 2022}

\begin{document}

\maketitle
\tableofcontents

\pagebreak

\section{Problem 1}
\[
	\begin{array}{l|l l}
		R \text{ is}        & \text{iff } & R^{-1} \text{ is}   \\
		\hline
		\text{total}        &             & \text{a surjection} \\
		\text{a function}   &             & \text{an injection} \\
		\text{a surjection} &             & \text{total}        \\
		\text{an injection} &             & \text{a function}   \\
		\text{a bijection}  &             & \text{a bijection}
	\end{array}
\]

\pagebreak

\section{Problem 2}
Let $A = \{a_0, a_1, ..., a_{n-1}\}$ be a set of size $n$, and and $B = \{b_0,b_1,...,b_{m-1}\}$ be a set of size $m$. Prove that
\begin{equation}\label{p39claim}
	|A \times B| = mn
\end{equation}
\begin{proof}
	Let
	\begin{align*}
		 & f((a_i, b_j)) \Coloneqq mi + j                            \\
		 & C \Coloneqq \Set{z \in \mathbb{N} | 0 \le z \le (mn - 1)}
	\end{align*}

	For every $c \in C$, there is exactly 1 $s \in A \times B$ such that $f(s) = c$. Therefore, $f$ is a bijection from $A \times B$ to $C$. From part 3 of Lemma 4.5.3 on page 116 of the book, it follows that
	\[
		|A \times B| = |C| = mn
	\]

	Hence, \eqref{p39claim} must be true.
\end{proof}

\pagebreak

\section{Problem 3}
\subsection{(a)}
\[
	|f(A)| \le |B|
\]

\subsection{(b)}
\[
	|A| \ge |B|
\]

\subsection{(c)}
\[
	|f(A)| = |B|
\]

\subsection{(d)}
\[
	|f(A)| = |A|
\]

\subsection{(e)}
\[
	|A| = |B|
\]

\pagebreak

\section{Problem 4}
Prove that
\begin{equation}\label{p40claim}
	|X| \ge |R(X)|
\end{equation}
\begin{proof}
	Let $n$ be the number of arrows that originate from an element of $X$.  Since $R$ is a function, at most 1 arrow originates from each member of $X$. Therefore,
	\begin{equation}\label{p40eq1}
		|X| \ge n
	\end{equation}

	From the definition of image, we know that $\forall r \in R(x)$ there exists an arrow originating from a member of $X$ that terminates at $r$. Therefore,
	\begin{equation}\label{p40eq2}
		|R(X)| \le n
	\end{equation}

	From \eqref{p40eq1} and \eqref{p40eq2}, it follows that
	\[
		|X| \ge n \ge |R(X)|
	\]

	Hence, \eqref{p40claim} must be true.

\end{proof}

\pagebreak

\section{Problem 5}
\subsection{(a)}
Prove that
\begin{equation}\label{p22aclaim}
	A \surj B \land B \surj C \implies A \surj C
\end{equation}
\begin{proof}
	\begin{align*}
		A \surj B \implies \exists \text{ a surjective function } f : A \rightarrow B \\
		B \surj C \implies \exists \text{ a surjective function } g : B \rightarrow C
	\end{align*}
	Since g is a surjective function, $\forall c \in C \; \exists b \in B.g(b) = c$. Since f is a surjective function, $\forall b \in B \; \exists a \in A.f(a) = b$. Therefore, $\forall c \in C \; \exists a \in A.g(f(a)) = c$. In other words, $g \circ f : A \rightarrow C$ is a surjective function. Hence, \eqref{p22aclaim} must be true.
\end{proof}

\subsection{(b)}
Prove that
\begin{equation}\label{p22bclaim}
	A \surj B \iff B \inj A
\end{equation}
\begin{proof}
	We will proceed by proving the following lemmas:
	\begin{align}
		\label{p22blemma1} A \surj B \implies B \inj A \\
		\label{p22blemma2} B \inj A \implies A \surj B
	\end{align}
	\begin{subproof}[Proof of \eqref{p22blemma1}]
		The proof is by contradiction. Suppose $A \surj B$ is true but $B \inj A$ is false.
		Then $\exists \text{ a surjective function } f : A \rightarrow B$.

		Suppose that $f^{-1} B \rightarrow A$ isn't total. Then $\exists b \in B$ such that no arrows of $f^{-1}$ originate from $b$. However, this would imply that no arrows of $f$ terminate at $b$.  Therefore, we've reached a contradiction where $f$ isn't surjective. Hence, $f^{-1}$ must be total.

		Suppose that $f^{-1} B \rightarrow A$ isn't injective. Then $\exists a \in A$ such that more than 1 arrow of $f^{-1}$ terminate at $a$. However, this would imply that more than 1 arrow of $f$ originates at $a$. Therefore, we've reached a contradiction where $f$ isn't a function. Hence, $f^{-1}$ must be injective.

		Since $f^{-1}$ is both total and injective, $f^{-1}$ is an injective total relation. Therefore, we've reached a contradiction where $B \inj A$ is true. Hence, \eqref{p22blemma1} must be true.
	\end{subproof}

	\begin{subproof}[Proof of \eqref{p22blemma2}]
		The proof is by contradiction. Suppose that $B \inj A$ is true but $A \surj B$ is false. Then $\exists \text{ an injective total relation } g : B \rightarrow A$.

		Suppose that $g^{-1} : A \rightarrow B$ isn't surjective. Then $\exists b \in B$ such that no arrows of $g^{-1}$ terminate at B. However, this would imply that no arrows of $g$ originate at $b$. Therefore, we've reached a contradiction where $g$ isn't total. Hence, $g^{-1}$ must be surjective.

		Suppose that $g^{-1}$ isn't a function. Then $\exists a \in A$ such that more than 1 arrow of $g^{-1}$ originates at $a$. However, this would imply that more than 1 arrow of $g$ terminates at $a$. Therefore, we've reached a contradiction where $g$ isn't injective. Hence, $g^{-1}$ must be a function.

		Since $g^{-1}$ is both surjective and a function, $g^{-1}$ is a surjective function. Therefore, we've reached a contradiction where $A \surj B$ is true. Hence, \eqref{p22blemma2} must be true.
	\end{subproof}

	We've proved that both \eqref{p22blemma1} and \eqref{p22blemma2} are true. Hence, \eqref{p22bclaim} must be true.
\end{proof}

\subsection{(c)}
Prove that
\begin{equation}\label{p22cclaim}
	A \inj B \land B \inj C \implies A \inj C
\end{equation}
\begin{proof}
	\begin{align*}
		 & A \inj B \land B \inj C   &                                  & \iff     \\
		 & B \surj A \land C \surj B & (\text{apply \eqref{p22bclaim}}) & \implies \\
		 & C \surj A                 & (\text{apply \eqref{p22aclaim}}) & \iff     \\
		 & A \inj C                  & (\text{apply \eqref{p22bclaim}})
	\end{align*}
	Hence, \eqref{p22cclaim} must be true.
\end{proof}

\subsection{(d)}
Prove that
\begin{equation}\label{p22dclaim1}
	A \inj B \iff \exists \text{ a total injective function f} : A \rightarrow B
\end{equation}
\begin{proof}
	From part 2 of Definition 4.5.2, we can rewrite \eqref{p22dclaim1} as
	\begin{align}
		 & \exists \text{ a total injective relation g} : A \rightarrow B & \iff \notag \\ &\exists \text{ a total injective function f} : A \rightarrow B& \label{p22dclaim2}
	\end{align}

	We will proceed by proving the following lemmas:
	\begin{align}
		 & \exists \text{ a total injective function f} : A \rightarrow B & \implies \notag \\ &\exists \text{ a total injective relation g} : A \rightarrow B& \label{p22dlemma1} \\
		 & \exists \text{ a total injective relation g} : A \rightarrow B & \implies \notag \\ &\exists \text{ a total injective function f} : A \rightarrow B& \label{p22dlemma2}
	\end{align}

	\begin{subproof}[Proof of Lemma \eqref{p22dlemma1}]
		\eqref{p22dlemma1} is trivially true since a function is a type of relation.  In other words, $f$ is also a relation.
	\end{subproof}

	\begin{subproof}[Proof of Lemma \eqref{p22dlemma2}]
		$f$ can be constructed from $g$ by eliminating arrows of $g$ until there is exactly one arrow originating from every $a \in A$.  Hence, \eqref{p22dlemma2} must be true.
	\end{subproof}

	We've proved that \eqref{p22dlemma1} and \eqref{p22dlemma2} are true. Hence, \eqref{p22dclaim2} must be true and \eqref{p22dclaim1} must be true.

\end{proof}

\end{document}