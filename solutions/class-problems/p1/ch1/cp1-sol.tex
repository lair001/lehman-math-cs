\documentclass{article}
\usepackage[utf8]{inputenc}
\usepackage{indentfirst, hyperref, gensymb, amsmath, amsthm, wasysym, amsfonts, mathtools, braket, amssymb}
\hypersetup{colorlinks,allcolors=blue,linktoc=all}

\title{In-Class Problem Solutions for Session 1}
\author{Samuel Lair}
\date{August 2022}

\begin{document}

\maketitle
\tableofcontents

\pagebreak

\section{Problem 1}
\subsection{(a)}
See the slides and lecture video for the details of this proof.
\subsection{(b)}
Again, see the slides and lecture video for the details of this proof.
\subsection{(c)}
If a = b, then our right triangle is isosceles and the four copies of our triangle completely cover a $c \times c$ square.  They also completely cover a $2a \times a$ rectangle.  I.e. the area of the square goes to zero. From this we can conclude:
\[c^2 = c \cdot c = 2a \cdot a = 2a^2 = a^2 + a^2 = a^2 + b^2\]
I.e. this proof of the Pythagorean Theorem still holds even when $a = b$.

You could also object to this proof by pointing out the case where $a > b$. However, you could simply swap the labels so that the longer leg is now labeled $b$ and the shorter leg is now labeled $a$. With this new labeling, $a < b$ and the proof proceeds as before.
\subsection{(d)}
Indeed, this \textit{Proof by Picture} makes a number of assumptions:
\begin{enumerate}
	\item All lines are straight.
	\item All angles are drawn to scale.
	\item The angles of a triangle sum to 180\degree.
	\item The angle opposite to $c$ is 90\degree.
	\item The four angles of a square are all 90\degree.
	\item The four sides of a square are all equal in length.
\end{enumerate}

Due to the difficulty in validating that a diagram is drawn correctly, algebraic proofs are preferable to pictorial proofs when available.

\pagebreak
\section{Problem 2}
\subsection{(a)}
The step $\sqrt{(-1)(-1)} = \sqrt{-1}\sqrt{-1}$ is incorrect. The short explanation is that although $i$ is defined such that $i^2 = -1$, $\sqrt{x}$ is undefined for $x < 0$. The longer explanation is that if we try to define $f(x) = \sqrt(x)$ for $x < 0$, we would need to let $f(x)$ map each real negative number to two imaginary numbers. Observe:
\[-i^2 = (-1)^2(i^2) = i^2 = -1\]
Therefore, $f(-1) = \pm i$. Indeed, if we accept that $\forall \; n < 0 \; \exists \; p > 0$ such that $n = -p$, $f(n) = \sqrt{-p} = \sqrt{-1}\sqrt{p} = \pm i \sqrt{p}$. From here, we conclude that $\sqrt{-1}\sqrt{-1}$ is equal to four different pairings of $-i$ and $i$: $-i \cdot i$, $i \cdot -i$, $i \cdot i$, and $-i \cdot -i$. These pairs evaluate to two distinct values: $\pm 1$. Therefore, in the step $\sqrt{(-1)(-1)} = \sqrt{-1}\sqrt{-1}$, the proof not only assumes that $\sqrt{(-1)(-1)} = 1$ but also that $\sqrt{(-1)(-1)} = -1$. Hence, the "proof" assumes what it is trying to prove and is "bogus".
\subsection{(b)}
If $1 = -1$:
\[1 \cdot \frac{1}{2} = -1 \cdot \frac{1}{2}\]
\[\frac{1}{2} = - \frac{1}{2}\]
\[\frac{1}{2} + \frac{3}{2} = - \frac{1}{2} + \frac{3}{2}\]
\[\frac{4}{2} = \frac{2}{2}\]
\[2 = 1\]
\subsection{(c)}
Prove that for positive real numbers $r$ and $s$, $\sqrt{rs} = \sqrt{r}\sqrt{s}$ where $\sqrt{rs}$ is the positive square root of $rs$, $\sqrt{r}$ is the positive square root of $r$ and $\sqrt{s}$ is the positive square root of $s$.
\begin{proof}
	We use proof by contradiction. Suppose the claim is false such that $\sqrt{rs} \neq \sqrt{r}\sqrt{s}$. Then
	\[
		\sqrt{rs}^2 \neq (\sqrt{r}\sqrt{s})^2
	\]
	\[
		rs \neq \sqrt{r}^2 \sqrt{s}^2
	\]
	\[
		rs \neq rs
	\]
	This is obviously a contradiction so the claim must be true. Hence, $\sqrt{rs} = \sqrt{r}\sqrt{s}$.
\end{proof}
\pagebreak
\section{Problem 3}
\subsection{(a)}
This bogus proof goes wrong in its second step. $\log_{10} (\frac{1}{2}) < 0$ so we must reverse the inequality we multiply both sides of the equation by this number. The corrected proof proceeds as follows:
\[
	3 > 2
\]
\[
	3 \log_{10} \left(\frac{1}{2}\right) < 2 \log_{10} \left(\frac{1}{2}\right)
\]
\[
	\log_{10} \left(\frac{1}{2}\right)^3 < \log_{10} \left(\frac{1}{2}\right)^2
\]
\[
	\left(\frac{1}{2}\right)^3 < \left(\frac{1}{2}\right)^2
\]
\[
	\frac{1}{8} < \frac{1}{4}
\]
This leads us to the correct claim that $\frac{1}{8} < \frac{1}{4}$.
\subsection{(b)}
The step $\$0.01 = (\$0.1)^2$ is incorrect because of the units. $0.01 = 0.1^2$ but $\$ \neq \$^2$ in the same way that $m \neq m^2$. I.e. the proof is conflating a "length" with an "area".
\subsection{(c)}
The "bug" in this proof is the third step where we subtract $b^2$ from both sides and conclude that $a^2 = b^2 = ab - b^2$. It isn't wrong per se but it is equivalent to stating that $0 = 0$, which locks us into the
solution $a = 0$. $a = 0$ is indeed a solution to $a = b$ but any real number is a solution to this equation as well.

\pagebreak

\section{Problem 4}
The bogus proof is objectionable because it starts out by assuming what it is trying to proof. I would fix it by reversing the order and filling in some additional details to clarify some of the steps.

Prove that $\frac{a + b}{2} \geq \sqrt{ab}$ for all nonnegative real numbers $a$ and $b$.

\begin{proof}
	We know that $(a - b)^2 \geq 0$ because $a - b$ is a real number and the square of a real number is never negative. We can proceed as follows:
	\[
		(a - b)^2 \geq 0
	\]
	\[
		a^2 - 2ab + b^2 \geq 0
	\]
	\[
		a^2 + 2ab + b^2 \geq 4ab
	\]
	\[
		(a + b)^2 \geq 4ab
	\]
	\[
		\text{Take the positive square root:}
	\]
	\[
		a + b \geq 2\sqrt{ab}
	\]
	\[
		\frac{a + b}{2} \geq \sqrt{ab}
	\]
	Hence, we conclude with our claim.
\end{proof}

\pagebreak

\section{Problem 5}
There are several ways in which the students' reasoning is flawed.

First of all, they are assuming that the timing of the quiz is the surprise. This isn't necessarily true. The surprise could lie in the content, the difficulty or some random detail like the color of the paper or style of the font. Everyone would be truly surprised if Serious Albert gave a quiz in rainbow colored Comic Sans!

Secondly, they are struggling with causality. You can adjust odds based on what you know about past events, but you can't do so based on what you speculate about future events.  If they reach the end of Thursday without a quiz, then they have a point that that the quiz will be given sometime on Friday. However, you can't use how you'd adjust the odds at the end of Thursday to predict what will happen on Thursday.

Thirdly, even though past events can narrow the timing of the quiz, past events never completely eliminate uncertainty about the timing of the quiz. If the quiz isn't given by the end of Thursday, then the students know that the quiz will occur on Friday. However, they still don't know when on Friday the quiz will occur. Hence, the potential for surprise remains.

\end{document}
